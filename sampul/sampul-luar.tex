\AddToShipoutPictureBG*{
  \AtPageLowerLeft{
    % Ubah nilai berikut jika posisi horizontal background tidak sesuai
    \hspace{-3.5mm}

    % Ubah nilai berikut jika posisi vertikal background tidak sesuai
    \raisebox{0mm}{
      \includegraphics[width=\paperwidth,height=\paperheight]{sampul/sampul-luar.png}
    }
  }
}

% Menyembunyikan nomor halaman
\thispagestyle{empty}

% Pengaturan margin untuk menyesuaikan konten sampul
\newgeometry{top=70mm,left=25mm,right=20mm,bottom=25mm}

\begin{flushleft}

  % Pengaturan jenis dan warna teks yang digunakan
  \sffamily\color{white}
  % Ubah kalimat berikut sesuai dengan judul topik kerja praktik
  \begin{center}
    \vspace{30ex}
    \noindent{\large \textbf{VIRTUAL MOUSE USING HAND GESTURE PYTHON}}
    \vspace{30ex}
  \end{center}

  \begin{adjustwidth}{-2mm}{}
    \begin{tabular}{lcp{0.7\linewidth}}
      % Ubah kalimat-kalimat berikut sesuai dengan nama dan NRP mahasiswa pertama
      \textbf{Argya Rijal Rafi} & & \textbf{NPM 1204062} \\
      % Ubah kalimat-kalimat berikut sesuai dengan nama dan NRP mahasiswa kedua
      \textbf{Guna Darma} & & \textbf{NPM 1204062} \\
    \end{tabular}
  \end{adjustwidth}
  \vspace{4ex}

  \noindent\textbf{Dosen Pembimbing} \\
  % Ubah kalimat berikut sesuai dengan nama dosen pembimbing
  \textbf{Rolly Awangga, S.T., M.T.}
  \vspace{10ex}

  % Ubah kalimat berikut sesuai dengan nama fakultas
  \textbf{Teknik Informatika} \\
  \textbf{Politeknik Pos Indonesia} \\
  % Ubah kalimat berikut sesuai dengan tempat dan tahun pembuatan buku
  \textbf{Bandung 2021}

\end{flushleft}

\restoregeometry
